\documentclass[a4paper,10pt]{article}


%error if activated%\RequirePackage[l2tabu, orthodox]{nag}			%Checks for obsolete LaTeX packages and outdated commands

\usepackage[T1]{fontenc}	%Correct accented characters and copy paste them, also not unexpected results with some characters like pipe
\usepackage[utf8]{inputenc}
\usepackage[spanish, es-tabla,es-noindentfirst]{babel}		%http://www.tex-tipografia.com/spanishopt.html
%\usepackage[spanish, es-tabla]{babel}		%http://www.tex-tipografia.com/spanishopt.html
				%es-noquotes]{babel}			si caracteres dan problemas
				%es-noshorthands]{babel}		si caracteres dan todavia problemas
				%es-noindentfirst]{babel}		para eliminar sangría
\usepackage[document]{ragged2e}	%centering
\usepackage{listings}			%highlight code
	\lstset{breaklines}			%line wrap for listings
	\newcommand{\lstlistinginput}{\lstinputlisting}		%alias
\usepackage{courier}		%listings better font
\usepackage[sorting=none]{biblatex}
	\addbibresource{bib.bib}
\usepackage{color}
\usepackage[table,xcdraw,usenames,dvipsnames]{xcolor}
%\usepackage{amsmath}		%https://www.sharelatex.com/learn/Aligning_equations_with_amsmath
%\usepackage{amsfonts}		%Extended set of fonts for use in mathematics
%\usepackage{afterpage}		%Commands specified in its argument are expanded after the current page is output. EG: \afterpage{\clearpage} and the current page will be filled up with text as usual, but then the \clearpage command will flush out all the floats before the next text page begins
\usepackage{graphicx}		%Introduces the \includegraphics command, which is needed for inserting figures
%\usepackage{microtype}		%Improves spacing (words and letters) and more stuff. Load after fonts, because is dependent on this font
%\usepackage{siunitx}		%SI units
%\usepackage{xspace}			%Decides whether to insert a space to replace one "eaten" by the command decoder
%\usepackage{todonotes}		%Mark things to do later
\usepackage{hyperref}		%\hyperref, \url and \href
%\usepackage[a4paper]{geometry}	%Margins without needing to remember the particular page dimensions commands. Eg [a4paper], [top=1in, bottom=1.25in, left=1.25in, right=1.25in]
%\usepackage{cleveref}		%LAST \usepackage in the preamble. If anything else modifies the referencing system (like amsmath) it all goes wrong

%\DisableLigatures{encoding = *, family = *}		%https://en.wikibooks.org/wiki/LaTeX/Text_Formatting#Ligatures
%\def\spanishoperators{}		%stuff like sin traduced to sen, max to máx, etc

\newcommand\realnumberstyle[1]{}

\makeatletter
\newcommand{\zebra}[3]{%
    {\realnumberstyle{#3}}%
    \begingroup
    \lst@basicstyle
    \ifodd\value{lstnumber}%
        \color{#1}%
    \else
        \color{#2}%
    \fi
        \rlap{\hspace*{\lst@numbersep}%
        \color@block{\linewidth}{\ht\strutbox}{\dp\strutbox}%
        }%
    \endgroup
}
\makeatother

%%%%%%%%%%%%%%%%%%%%%%%%%		MORE	ALIASES		%%%%%%%%%%%%%%%%%%%%%%%%%

\newcommand{\lineh}{\rule{\textwidth}{1pt}\hfill\break}
\newcommand{\linej}{\hfill\break}



%%%%%%%%%%%%%%%%%%%%%%%%%		END BASIC CONF		%%%%%%%%%%%%%%%%%%%%%%%%%



%\definecolor{mygreen}{rgb}{0,0.6,0}
%\definecolor{mygray}{rgb}{0.5,0.5,0.5}


%\renewcommand{\thesubsection}{\thesection.\alph{subsection}}		%change subsection style to letters



%https://en.wikibooks.org/wiki/LaTeX/Source_Code_Listings


%\lstdefinestyle{html}{
	%language=HTML,
%}

%\lstdefinestyle{sql}{
	%language=sql,
%}

%\lstdefinestyle{XML}{
	%language=XML,
%}

%\lstdefinestyle{Java}{
	%language=Java,
%}

%\lstdefinestyle{C}{
	%language=C,		%for syntax highlighting
	%emph={int,char,double,float,unsigned},
	%emphstyle=\color{green},
%}

%\lstdefinestyle{sh}{
	%language=sh,
	%%directivestyle=\color{gray},
	%emph={local, export, sudo },
	%emphstyle=\color{green}\bfseries,
%}

%\lstdefinelanguage{JavaScript}{
	%keywords={typeof, new, true, false, catch, function, return, null, catch, switch, var, if, in, while, do, else, case, break},
	%ndkeywords={class, export, boolean, throw, implements, import, this},
	%ndkeywordstyle=\color{gray}\bfseries,
	%identifierstyle=\color{black},
	%sensitive=false,
	%comment=[l]{//},
	%morecomment=[s]{/*}{*/},
	%morestring=[b]',
	%morestring=[b]"
%}
%\lstdefinestyle{JavaScript}{
	%language=JavaScript,
%}

%\lstdefinelanguage{CSS}{
	%keywords={background-image:,border:,border-top:,border-left:,border-right:,border-bottom:,border-radius:,border-style:,clear:,color:,margin:,margin-top:,margin-left:,margin-right:,margin-bottom:,padding:,flex-direction:,flex-grow:,float:,font:,font-size:,weight:,display:,position:,top:,left:,right:,bottom:,list:,list-style:,style:,border:,size:,space:,min:,width:,text-align:,text-shadow:, transition:, transform:, transition-property:, transition-duration:, transition-timing-function:},
	%sensitive=true,
	%morecomment=[l]{//},
	%morecomment=[s]{/*}{*/},
	%morestring=[b]',
	%morestring=[b]",
	%alsoletter={:},
	%alsodigit={-}
%}
%\lstdefinestyle{CSS}{
	%language=CSS,
%}

%\lstdefinelanguage{HTML5}{
		%language=html,
		%sensitive=true,
		%alsoletter={<>=-},
		%otherkeywords={
		%% HTML tags
		%<html>, <head>, <title>, </title>, <meta, />, </head>, <body>,
		%<canvas, \/canvas>, <script>, </script>, </body>, </html>, <!, html>, <style>, </style>, ><
		%},
		%ndkeywords={
		%% General
		%=,
		%% HTML attributes
		%charset=, id=, width=, height=,
		%% CSS properties
		%border:, transform:, -moz-transform:, transition-duration:, transition-property:, transition-timing-function:
		%},
		%morecomment=[s]{<!--}{-->},
		%tag=[s]
%}
%\lstdefinestyle{HTML5}{
	%language=HTML5,
%}


%\lstset{
	%basicstyle=\footnotesize\ttfamily,
	%numbers=left,
	%numberstyle=\tiny\zebra{green!07}{white},
	%frame=tb,
	%tabsize=4,
	%columns=fixed,
	%showstringspaces=false,
	%showtabs=false,
	%keepspaces,
	%commentstyle=\color{red},
	%stringstyle=\color{orange},
	%extendedchars=true,
	%literate={á}{{\'a}}1
	%{é}{{\'e}}1
	%%{í}{{\'{\i}}}1
	%{í}{{\'i}}1
	%{ó}{{\'o}}1
	%{ú}{{\'u}}1
	%{Á}{{\'A}}1
	%{É}{{\'E}}1
	%{Í}{{\'I}}1
	%{Ó}{{\'O}}1
	%{Ú}{{\'U}}1
	%%{ü}{{\"u}}1
	%%{Ü}{{\"U}}1
	%{ñ}{{\~n}}1
	%{Ñ}{{\~N}}1
	%{¿}{{?``}}1
	%{¡}{{!``}}1,
	%keywordstyle=\color{blue}\bfseries
%}



%\lstdefinestyle{file}{
	%basicstyle=\footnotesize\ttfamily,
	%frame=tblr,
	%title=\lstname
%}




%atm general file listing with
%\lstinputlisting[style=file]{}


%%%%%%%%%%%%%%%%%%%%%%%%%		END CURRENT DOCUMENT CONF		%%%%%%%%%%%%%%%%%%%%%%%%%




\author{
	Andrés Santiago Gómez Vidal
		\\
		\textit{andressantiago.gomez@}			%TODO
}

\title{}
\date{\today}


\makeatletter
\hypersetup{pdftitle={\@title},pdfauthor={\@author}}
\makeatother



\begin{document}

\maketitle
\pagenumbering{gobble}		%gobble=no numbers, arabic=arabic numbers, roman=roman numbers
\newpage

\pagenumbering{arabic}		%gobble=no numbers, arabic=arabic numbers, roman=roman numbers
\tableofcontents
\newpage





\section{}



%\subsection{}
%\subsubsection{}

%\paragraph{}
%\subparagraph{}



%\section{Bibliografía}
%\printbibliography[type=book,heading=subbibliography,title={Libros}]
%\printbibliography[type=article,heading=subbibliography,title={Artículos}]
%\printbibliography[type=online,heading=subbibliography,title={Referencias web}]
%\printbibliography[nottype=book,nottype=online,nottype=article,heading=subbibliography,title={Otras fuentes}]


\end{document}






%%%%%%%%%%%%%%%%%%%%%%%%%		STUFF AND EXAMPLES		%%%%%%%%%%%%%%%%%%%%%%%%%

%\begin{figure}[H] 	%recommended!

%\begin{figure}		%[h] (here), [t] (top), [b] (bottom), [p] (page): on an extra page, [!] (override): will force the specified location
	%\includegraphics[width=\linewidth]{}
	%\caption{}
	%\label{fig:}
%\end{figure}


%\hfill\break \centerline{$\downarrow$} \hfill\break



%https://en.wikibooks.org/wiki/LaTeX/Text_Formatting
	%\linespread{factor}

	%\textsubscript{text}
	%\textsuperscript{text}

	%\begin{doublespace}		\end{doublespace}
	%\begin{spacing}{2.5}		\end{spacing}

	%\hfill		%You can insert a horizontal stretched space with \hfill in a line so that the rest gets "pushed" toward the right margin
	%\vfill		%Similarly you can insert vertical stretched space with \vfill

	%\mbox{text}		%One or more words can be kept together on the one line
	%\fbox is similar to \mbox, but in addition there will be a visible box drawn around the content

	%\usepackage{setspace}
		%\singlespacing
		%\onehalfspacing
		%\doublespacing
		%\setstretch{1.1}

	%\ldots		%You cannot enter ‘ellipsis’=... by just typing three dots, as the spacing would be wrong. Instead, there is that special command

	%Words that use slash marks, such as "input/output" should be typeset as "input\slash output", which allow the line to "break" after the slash mark (if needed). The use of the / character in LaTeX should be restricted to units, such as "mm/year", which should not be broken over multiple lines

	%latex gives formating trouble with character -
		%To avoid hyphenation altogether, the penalty for hyphenation can be set to an extreme value:	\hyphenpenalty=100000
		%LaTeX knows four kinds of dashes: a hyphen (-), en dash (–), em dash (—), or a minus sign (−). You can access three of them with different numbers of consecutive dashes. The fourth sign is actually not a dash at all—it is the mathematical minus sign
			%Hyphen: daughter-in-law, X-rated\\
			%En dash: pages 13--67\\
			%Em dash: yes---or no? \\
			%Minus sign: $0$, $1$ and $-1$
	%With spanish babel:
		%Sometimes for formatting reasons some words have to be broken up in syllables separated by a - (hyphen) to continue the word in a new line. For example, matemáticas could become mate-máticas. The package babel, whose usage was described in the previous section, usually does a good job breaking up the words correctly, but if this is not the case you can use a couple of commands in your preamble.

		 %\usepackage{hyphenat}
		 %\hyphenation{mate-máti-cas recu-perar}

		%The first command will import the package hyphenat and the second line is a list of space-separated words with defined hyphenation rules. On the other side, if you want a word not to be broken automatically, use the {\nobreak word} command within your document


%Quotes commands		%https://es.sharelatex.com/learn/Typesetting_quotations




%http://www.ottobib.com/


%https://tex.stackexchange.com/questions/33967/highlighting-extremal-values-in-table
